% This file is part of documentation for ISAMPP
% 2018 - 2022, Ivan Kmeťo
%
% CC0 1.0 Universal (CC0 1.0) Public Domain Dedication
% https://creativecommons.org/publicdomain/zero/1.0/


\documentclass{article}
\usepackage[utf8]{inputenc}
\usepackage{graphicx}

\usepackage[left=1in,right=1in,top=1in,bottom=.8in]{geometry}
\usepackage[dvipsnames]{xcolor}
\ifpdf
\usepackage[colorlinks,linkcolor=blue,urlcolor=blue,citecolor=blue,
plainpages=false,pdfpagelabels,breaklinks]{hyperref}
\else
\usepackage[colorlinks,linkcolor=blue,urlcolor=blue,citecolor=blue,
plainpages=false,pdfpagelabels,linktocpage]{hyperref}
\fi


\title{Includes for San Andreas Multiplayer Plus\\Version 1.9, Documentation}
\author{Ivan Kmeťo}
\date{July 2022}


\begin{document}

\maketitle
\begin{center}
\includegraphics[width = 40mm]{logo/isampp_logo_big.png}
\end{center}

\newpage
\tableofcontents

\newpage
\section{Changes in ISAMPP 1.9}
\textbf{ISAMPP Library:}
\\- Completed object list for props2.ide \textit{(ide\_props2.inc)}
\\- New objects defined \textit{(i\_objects.inc)}
\\- New sounds defined \textit{(i\_soundids.inc)}
\\- SND\_GYM\_ sounds use prefix SND\_GYMGUY\_ \textit{(i\_soundids.inc)}
\\- SND\_CROUPIER\_ sounds use prefix SND\_CROUPIERF\_ \textit{(i\_soundids.inc)}
\\- SND\_CEILING\_VENT\_LAND renamed to SND\_VENT\_DROP \textit{(i\_soundids.inc)}
\\- SND\_HIT\_CAR renamed to SND\_CAR\_DENT \textit{(i\_soundids.inc)}
\\- PICKUP\_GIFTSMALL renamed to PICKUP\_GUNBOXSMALL \textit{(i\_pickupids.inc)}
\\- PICKUP\_GIFTBIG renamed to PICKUP\_GUNBOXBIG \textit{(i\_pickupids.inc)}
\\- PICKUP\_LANDMINE renamed to PICKUP\_MINE \textit{(i\_pickupids.inc)}
\\- PICKUP\_HEART renamed to PICKUP\_HEALTH \textit{(i\_pickupids.inc)}
\\- PICKUP\_PILL renamed to PICKUP\_ADRENALINE \textit{(i\_pickupids.inc)}
\\- PICKUP\_ARMOR renamed to PICKUP\_BODYARMOUR \textit{(i\_pickupids.inc)}
\\- PICKUP\_GTA3LOGO renamed to PICKUP\_BONUS \textit{(i\_pickupids.inc)}
\\- PICKUP\_RAMPAGESSKULL renamed to PICKUP\_KILLFRENZY \textit{(i\_pickupids.inc)}
\\- PICKUP\_PROPERTYBLUE renamed to PICKUP\_PROPERTY\_LOCKED \textit{(i\_pickupids.inc)}
\\- PICKUP\_PROPERTYGRN renamed to PICKUP\_PROPERTY\_FSALE \textit{(i\_pickupids.inc)}
\\- PICKUP\_DOLLAR renamed to PICKUP\_BIGDOLLAR \textit{(i\_pickupids.inc)}
\\- PICKUP\_BLSHIRT renamed to PICKUP\_CLOTHES \textit{(i\_pickupids.inc)}
\\- PICKUP\_DRUGS renamed to PICKUP\_CRAIGPACKAGE \textit{(i\_pickupids.inc)}
\\- PICKUP\_RAMPAGEMSKULL renamed to PICKUP\_KILLFRENZY2 \textit{(i\_pickupids.inc)}
\\- PICKUP\_YELLOWBOTTLE renamed to PICKUP\_LOTION \textit{(i\_pickupids.inc)}
\\- PICKUP\_RUSTYM4 renamed to PICKUP\_CJM16 \textit{(i\_pickupids.inc)}
\\- PICKUP\_RUSTYSNIPER renamed to PICKUP\_CJPSG1 \textit{(i\_pickupids.inc)}
\\- PICKUP\_SQUARESHOVEL renamed to PICKUP\_CJSHOVEL \textit{(i\_pickupids.inc)}
\\- PICKUP\_STRIPEDQUESTIONM renamed to PICKUP\_NOMODELFILE \textit{(i\_pickupids.inc)}
\\- PICKUP\_REDARROWDOWN renamed to PICKUP\_ARROWTYPE4 \textit{(i\_pickupids.inc)}
\\- PICKUP\_YELLOWENMARKER renamed to PICKUP\_ENEXMARKER3 \textit{(i\_pickupids.inc)}
\\- PICKUP\_LANDMINE2 renamed to PICKUP\_LANDMINE1 \textit{(i\_pickupids.inc)}
\\- Updated documentation
\bigskip
\\\textbf{ISAMPP Sandbox Game Mode:}
\\- Money and Health pickups now work 
\bigskip
\\\textit{Changelogs for previous versions of ISAMPP are located in folder "docs/changelogs"}

\newpage
\section{Introduction}
San Andreas Multiplayer Includes Library+ or ISAMPP \textit{(Includes for San Andreas Multiplayer Plus)} is library of include files for San Andreas Multiplayer Mod SA-MP. Purpose of ISAMPP is to make development of game modes for SA-MP much easier. SA-MP by default uses in most cases numeric identifiers which are hard to remember and without enough experience you have to use SA-MP wiki a lot. Word identifiers are much more easier to remember, they make much more sense (especially if you are familiar with modding the game Grand Theft Auto: San Andreas) and may also have positive impact on your workflow. Except re-definitions of numeric identifiers, ISAMPP also contains other useful libraries - such as library of location names with coordinates, vehicle names list, legacy scripting functions, or few related custom stock functions. ISAMPP is available with sandbox-styled game mode in which you can see everything implemented.

\textit{\\ISAMPP is not part of San Andreas Multiplayer mod (SA-MP) and it is not affiliated with Rockstar Games, Rockstar North or Take-Two Interactive Software Inc.}
\textit{\\Grand Theft Auto and Grand Theft Auto: San Andreas are registered trademarks of Take-Two Interactive Software Inc.}

\textit{\\ISAMPP versions 1.3 or newer should be considered public domain.}
\\\url{https://creativecommons.org/publicdomain/zero/1.0/}
\bigskip
\\\textit{Older versions were published under MIT License.}
\\\url{http://opensource.org/licenses/mit-license}

\section{Installation}
Copy contents of \textit{include} folder to \textit{"[SA-MP Server folder]/include"} and also to \textit{"include"} folder for Pawno (by default \textit{"[SA-MP Server folder]/pawno/include"}).
\\In your game mode file you can include ISAMPP header for all its contents:
\begin{verbatim}
#include <i_sampp>
\end{verbatim}
or include header files as you wish separately:
\begin{verbatim}
#include <i_sampp/i_bodyparts>
#include <i_sampp/i_boneids>
#include <i_sampp/i_cammode>
#include <i_sampp/i_carcols>
#include <i_sampp/i_carmods>
#include <i_sampp/i_colorlist>
#include <i_sampp/i_crimes>
#include <i_sampp/i_explosions>
#include <i_sampp/i_funcl>
#include <i_sampp/i_iconids>
#include <i_sampp/i_locationids>
#include <i_sampp/i_objects>
#include <i_sampp/i_paintjob>
#include <i_sampp/i_pickupids>
#include <i_sampp/i_pickuptypes>
#include <i_sampp/i_skinids>
#include <i_sampp/i_soundids>
#include <i_sampp/i_textstyle>
#include <i_sampp/i_vehhealth>
#include <i_sampp/i_vehids>
#include <i_sampp/i_weaponids>
#include <i_sampp/i_weatherids>
\end{verbatim}
If you wish to run included sandbox game mode, you have to add file testmpp.pwn to your \textit{"gamemodes"} folder and compile it. Alternatively, you can use pre-compiled file testmpp.amx.

\newpage
\section{Includes}
Every ISAMPP include file starts with prefix \textit{i\_}. Please keep in mind that ISAMPP may not be the only set of include files using this prefix.
\bigskip
\\\textit{bodyparts} - List of available player/npc body part identifiers
\\\textit{boneids} - List of player bone identifiers
\\\textit{cammode} - List of known camera modes
\\\textit{carcols} - List of definitions for all available vehicle colors
\\\textit{carmods} - List of all available components for vehicle customization
\\\textit{colorlist} - List of color definitions
\\\textit{crimes} - List of crime reports
\\\textit{explosions} - List of available types of explosions
\\\textit{funcl} - Legacy functions
\\\textit{iconids} - List of map icon identifiers
\\\textit{locationids} - List of location definitions with coordinates, interior identifiers and names
\\\textit{objects}- List of game objects/models
\\\textit{paintjob} - List of definitions for all available vehicle paintjobs
\\\textit{pickupids} - List of definitions for pickup identifiers
\\\textit{pickuptypes} - List of definitions for pickup types
\\\textit{skinids} - List of definitions for character skin/model identifiers
\\\textit{soundids} - List of definitions for sound identifiers
\\\textit{textstyle} - List of definitions for GameText styles
\\\textit{vehhealth} - Vehicle Health configurations
\\\textit{vehids} - List of definitions for all available vehicles
\\\textit{weaponids} - List of definitions for weapon identifiers, sorted by weapon slot IDs
\\\textit{weatherids} - List of definitions for weather identifiers


\newpage
\section{Body Parts}
\begin{sloppypar}
Body part identifiers are included as macro defines with prefix \textit{BODYPART\_}. Body parts can be used in functions \textit{OnPlayerGiveDamage}, \textit{OnPlayerTakeDamage} and \textit{OnPlayerGiveDamageActor}. It is unknown if IDs 0, 1 and 2 have any use. You can test this feature in sandbox game mode by shooting Actor NPC.
\end{sloppypar}
\bigskip
\noindent\begin{tabular}{ |l|c|l| } 
\hline
Identifier & Value & Body Part \\
\hline
BODYPART\_TORSO & 3 & Torso \\ 
BODYPART\_GROIN & 4 & Groin \\
BODYPART\_LEFTARM & 5 & Left Arm \\
BODYPART\_RIGHTARM & 6 & Right Arm \\
BODYPART\_LEFTLEG & 7 & Left Leg \\
BODYPART\_RIGHTLEG & 8 & Right Leg \\
BODYPART\_HEAD & 9 & Head \\
\hline
\end{tabular}
\bigskip
\\Resources: \url{https://open.mp/docs/scripting/resources/bodyparts}

\section{Bone Identifiers}
\begin{sloppypar}
Bone identifiers with prefix \textit{BONE\_} can be used for attaching objects to specific parts of player model/skin with function \textit{SetPlayerAttachedObject}.
\end{sloppypar}
\bigskip
\noindent\begin{tabular}{ |l|c|l| } 
\hline
Identifier & Value & Bone \\
\hline
BONE\_SPINE & 1 & Spine \\
BONE\_HEAD & 2 & Head \\
BONE\_LEFTUPPERARM & 3 & Left Upper Arm \\
BONE\_RIGHTUPPERARM & 4 & Right Upper Arm \\
BONE\_LEFTHAND & 5 & Left Hand \\
BONE\_RIGHTHAND & 6 & Right Hand \\
BONE\_LEFTTHIGH & 7 & Left Thigh \\
BONE\_RIGHTTHIGH & 8 & Right Thigh \\
BONE\_LEFTFOOT & 9 & Left Foot \\
BONE\_RIGHTFOOT & 10 & Right Foot \\
BONE\_RIGHTCALF & 11 & Right Calf \\
BONE\_LEFTCALF & 12 & Left Calf \\
BONE\_LEFTFOREARM & 13 & Left Forearm \\
BONE\_RIGHTFOREARM & 14 & Right Forearm \\
BONE\_LEFTSHOULDER & 15 & Left Clavicle (shoulder) \\
BONE\_RIGHTSHOULDER & 16 & Right Clavicle (shoulder) \\
BONE\_NECK & 17 & Neck \\
BONE\_JAW & 18 & Jaw \\
\hline
\end{tabular}
\bigskip
\\\textbf{Example of implementation:}
\begin{verbatim}
public OnPlayerSpawn(playerid)
{
    //Give player green hat on spawn
    SetPlayerAttachedObject(playerid, BONE_LEFTUPPERARM, 19487, 2, 0.101, 
    -0.0, 0.0, 5.50, 84.60, 83.7, 1.0, 1.0, 1.0, 0xFF00FF00);

    return 1;
}
\end{verbatim}
\bigskip
Resources: \url{https://open.mp/docs/scripting/resources/boneid}


\newpage
\section{Camera Modes}
\begin{sloppypar}
List of known camera modes to be used as returned value of function \textit{GetPlayerCameraMode(playerid)}.\\In ISAMPP sandbox game mode you can output more information about active camera mode with command \textbf{/cameramode}
\end{sloppypar}
\bigskip
\noindent\begin{tabular}{ |l|c|l| } 
\hline
Identifier & Value & Description \\
\hline
CAMMODE\_TRAIN & 3 & Train/tram camera \\
CAMMODE\_FOLLOWPED & 4 & Follow player camera \\
CAMMODE\_SNIPER & 7 & Sniper rifle aiming \\
CAMMODE\_ROCKETAIM & 8 & Rocket launcher aiming \\
CAMMODE\_FIXED & 15 & Fixed camera (non-moving) \\
CAMMODE\_VEHICLEFRONT & 16 & Vehicle front camera, bike side camera \\
CAMMODE\_VEHICLEDEFAULT & 18 & Normal vehicle camera, several variable distances \\
CAMMODE\_BOATDEFAULT & 22 & Normal boat camera \\
CAMMODE\_WEAPONAIM & 46 & Weapon aiming camera \\
CAMMODE\_ROCKETAIM2 & 51 & Heat-seeking Rocket Launcher aiming camera \\
CAMMODE\_WEAPONAIM2 & 53 & Aiming any other weapon \\
CAMMODE\_VEHICLEPASSENGER & 55 & Vehicle passenger drive-by camera \\
CAMMODE\_HELICHASE & 56 & Chase camera: helicopter/bird view \\
CAMMODE\_GROUNDCHASE & 57 & Chase camera: ground camera, zooms in very quickly \\
CAMMODE\_FLYBYCHASE & 58 & Chase camera: horizontal flyby past vehicle \\
CAMMODE\_GROUNDCHASE2 & 59 & Chase camera (air vehicles): looking up \\
CAMMODE\_FLYBYCHASE2 & 62 & Chase camera (air vehicles): vertical flyby \\
CAMMODE\_FLYBYCHASE3 & 63 & Chase camera (air vehicles): horizontal flyby \\
CAMMODE\_PILOTCHASE & 64 & Chase camera (air vehicles): camera focused on pilot \\
\hline
\end{tabular}
\bigskip
\\Resources: \url{https://open.mp/docs/scripting/resources/cameramodes}


\section{Color List}
Color List contains definitions of some few common colors in two formats. Primary or Main color definitions have prefix \textit{COLOR\_} and are defined in 0xRRGGBBAA format. Secondary or String color definitions with prefix \textit{SCOL\_} are defined in more common hex format "\{RRGGBB\}" without the alpha color transparency values. The difference of these two formats can be explained by this example:
\begin{verbatim}
SendClientMessage(playerid, COLOR_RED, "Hello "SCOL_BLUE"World");
\end{verbatim}
\begin{sloppypar}
\noindent where the output will print in game client message box string of text in this format:
\textbf{\color{red}Hello \color{blue}World}
\end{sloppypar}
\bigskip
Secondary or String colors can be used directly in strings while Primary/Main colors would be used mainly as separate parameters in SA-MP functions. To see how every color looks in game, use in included sandbox mode for ISAMPP command \textit{/maincols [0-14]} or \textit{/stringcols}. Screenshot of color strings printed out in client message box is located in \textit{"docs/images/i\_colorlist.png"}.
\bigskip
\\Resources: \url{https://open.mp/docs/scripting/resources/colorslist}


\newpage
\section{Crime Reports}
\begin{sloppypar}
List of crime report identifiers starting with prefix \textit{CRIME\_} followed by ten-code. These macros can be used with function \textit{PlayCrimeReportForPlayer(playerid, suspectid, \textbf{crimeid});}
\end{sloppypar}
\bigskip
\noindent\begin{tabular}{ |l|c|l| } 
\hline
Identifier & Value & Description \\
\hline
CRIME\_10\_71 & 3 & 10-71 Advise nature of fire \\
CRIME\_10\_37\_1 & 4 & 10-37 Investigate suspicious vehicle \\
CRIME\_10\_81\_1 & 5 & 10-81 Breathalyzer Report \\
CRIME\_10\_24 & 6 & 10-24 Assignment Completed \\
CRIME\_10\_21\_1 & 7 & 10-21 Call () by phone \\
CRIME\_10\_21\_2 & 8 & 10-21 Call () by phone \\
CRIME\_10\_21\_3 & 9 & 10-21 Call () by phone \\
CRIME\_10\_17 & 10 & 10-17 Meet Complainant \\
CRIME\_10\_81\_2 & 11 & 10-81 Breathalyzer Report \\
CRIME\_10\_91\_1 & 12 & 10-91 Pick up prisoner/subject \\
CRIME\_10\_28\_1 & 13 & 10-28 Vehicle registration information \\
CRIME\_10\_81\_3 & 14 & 10-81 Breathalyzer Report \\
CRIME\_10\_28\_2 & 15 & 10-28 Vehicle registration information \\
CRIME\_10\_91\_2 & 16 & 10-91 Pick up prisoner/subject \\
CRIME\_10\_34 & 17 & 10-34 Riot \\
CRIME\_10\_37\_2 & 18 & 10-37 Investigate suspicious vehicle \\
CRIME\_10\_81\_4 & 19 & 10-81 Breathalyzer Report \\
CRIME\_10\_7\_1 & 21 & 10-7 Out of service \\
CRIME\_10\_7\_2 & 22 & 10-7 Out of service \\
\hline
\end{tabular}
\bigskip
\\\textbf{Example of implementation:}
\begin{verbatim}
public OnPlayerCommandText(playerid, cmdtext[])
{
    if (strcmp("/crime", cmdtext, true, 20) == 0) {
        PlayCrimeReportForPlayer(playerid, playerid, CRIME_10_71);
        return 1;
    }

    return 0;
}
\end{verbatim}
\bigskip
Resources: \url{https://open.mp/docs/scripting/resources/crimelist}


\section{Explosions}
There are 14 types of available explosions defined as macros with prefix \textit{EXPLOSION\_} and you can test every explosion with in-game command \textbf{/explosion [ID]}. These identifiers can be used with both functions \textit{CreateExplosion} or \textit{CreateExplosionForPlayer}.
\bigskip
\\Resources: \url{https://open.mp/docs/scripting/resources/explosionlist}


\newpage
\section{GameText Styles}
\begin{sloppypar}
List of definitions for SA-MP function \textit{GameTextForPlayer(playerid, const string[], time, \textbf{style})}. Some styles might not work properly or could crash the game.
\end{sloppypar}
\bigskip
\noindent\begin{tabular}{ |l|c|l| } 
\hline
Identifier & Value & Notes \\
\hline
GMTEXT\_STYLE\_PRICEDOWN & 0 & Appears for 9 seconds \\
GMTEXT\_STYLE\_RPRICEDOWN & 1 & Fades out after 8 seconds \\
GMTEXT\_STYLE\_SA & 2 & Does not disappear until player respawns \\
GMTEXT\_STYLE\_SLIM1 & 3 & San Andreas specific font \\
GMTEXT\_STYLE\_SLIM2 & 4 & San Andreas specific font \\
GMTEXT\_STYLE\_SLIMW & 5 & Displays for 3 seconds \\
GMTEXT\_STYLE\_BPRICEDOWN & 6 & Blue Pricedown font in middle of screen \\
GMTEXT\_STYLE\_VEHNAME & 7 & SA vehicle names (fixes.inc) \\
GMTEXT\_STYLE\_LOCATION & 8 & SA location names (fixes.inc) \\
GMTEXT\_STYLE\_RADIO & 9 & SA selected radio station names (fixes.inc) \\
GMTEXT\_STYLE\_RADIOW & 10 & SA switching radio station names (fixes.inc) \\
GMTEXT\_STYLE\_PMONEY & 11 & SA positive money (fixes.inc) \\
GMTEXT\_STYLE\_NMONEY & 12 & SA negative money (fixes.inc) \\
GMTEXT\_STYLE\_STUNT & 13 & SA stunt bonuses (fixes.inc) \\
GMTEXT\_STYLE\_CLOCK & 14 & SA in-game clock (fixes.inc) \\
GMTEXT\_STYLE\_NOTIFICATION & 15 & SA notification popup (fixes.inc) \\
\hline
\end{tabular}
\bigskip
\\fixes.inc: \url{https://github.com/pawn-lang/sa-mp-fixes}
\bigskip
\\Resources: \url{https://open.mp/docs/scripting/resources/gametextstyles}


\section{Location Identifiers}
\begin{sloppypar}
Locations, in-game interiors and exteriors, are stored in file \textit{i\_locationids.inc} as set of macro definitions for identifiers and three arrays. Array \textit{locCoords} is storing float values of XYZ position on the map and float value of PlayerFacingAngle. By default, your PlayerInterior value is set to 0 whenever you are "outside" on the map. When player enters the interior, in order to load the interior you have to change PlayerInterior value accordingly. This is what array \textit{locID} includes. Third array, \textit{locName}, has defined strings of names of particular locations so this information can be displayed to player if needed.
\end{sloppypar}
\bigskip
\noindent \textit{\textbf{Note:} If you wish to add new location to this list, you have to do it in correct order. Not every location has proper collision boxes because it was used only in cutscene, so in certain interiors you may fall through ground.}
\bigskip
\\Resources: \url{https://open.mp/docs/scripting/resources/interiorids}


\section{Map Icon Identifiers}
There are in total 64 map icons included as macro defines with prefix \textit{ICON\_}.
\bigskip
\\Resources: \url{https://open.mp/docs/scripting/resources/mapicons}


\section{Objects}
Available game objects are in order by IDE file and ID number. Every game object is listed with prefix \textit{OBJ\_} followed by its name of DFF file.
\bigskip
\\Resources: \url{https://open.mp/docs/scripting/resources/samp_objects}


\newpage
\section{Pickup Identifiers}
Pickup identifiers are included as macro defines with prefix \textit{PICKUP\_}. Every pickup defined in ISAMPP is placed on map in ISAMPP sandbox game mode. Weapon pickups use combined prefix \textit{PICKUP\_WEAP\_} for easier use.
\bigskip
\\Resources: \url{https://open.mp/docs/scripting/resources/pickupids}


\section{Pickup Types}
Pickup types specify behavior of created pickups in game modes. Every pickup type available in SA-MP is listed with prefix \textit{PICKUP\_TYPE\_}
\bigskip
\\Resources: \url{https://open.mp/docs/scripting/resources/pickuptypes}


\section{Skin/Playermodel Identifiers}
Skins are included as macro defines with prefix \textit{SKIN\_}, followed by name of model in game. With exception of few character models \textit{(mainly main story characters and other special characters)} where the name of model is obvious \textit{(f.e. emmet = Emmet, etc.)}, character models are named in this or similar format: prefix \textit{(f.e. if is located only in certain area)}, race, gender, age, suffix. In this context BMYRI = \textbf{B}lack, \textbf{M}ale, \textbf{Y}oung, \textbf{RI}ch. SBFYS = (\textbf{S}an \textbf{F}ransisco) \textbf{B}lack, \textbf{F}emale, \textbf{Y}oung, \textbf{S}treet, etc. This format is very helpful in contrast with plain number identifiers SA-MP uses by default.
\bigskip
\\Resources: \url{https://open.mp/docs/scripting/resources/skins}


\section{Sound Identifiers}
Sounds meant to be played with function \textit{PlayerPlaySound} are defined with prefix \textit{SND\_} in file i\_soundids.inc. Please note that this list of available sounds may be incomplete and some sound loops might cause the game client to crash.
\bigskip
\\Resources: \url{https://open.mp/docs/scripting/resources/sound-ids}


\section{Vehicle Components}
List of all available components for vehicle customization, starting with prefix \textit{CARMOD\_}\\Can be used with function \textit{AddVehicleComponent(vehicleid, \textbf{componentid});}
\bigskip
\\Resources: \url{https://open.mp/docs/scripting/resources/carcomponentid}


\section{Vehicle Identifiers}
Vehicle identifiers are included as macro defines with prefix \textit{VEH\_} followed by the name of vehicle in game with capital letters \textit{(motorcycles, bicycles, boats, planes, helicopters, trailers, trains, etc. are included in this format as well)}. For better orientation in list, defines are sorted by vehicle type.
\bigskip
\\Resources: \url{https://open.mp/docs/scripting/resources/vehicleid}


\newpage
\section{Vehicle Health}
\begin{sloppypar}
Vehicle Health configurations are included as macro defines with prefix \textit{VEH\_HEALTH\_} followed by the desired pre-defined identifier. These values are only related to engine condition and do not change visual damage of vehicle model.
\end{sloppypar}
\bigskip
\noindent\begin{tabular}{ |l|c|l| } 
\hline
Identifier & Exact Value & Description \\
\hline
VEH\_HEALTH\_FULL & 1000 & Full vehicle health \\ 
VEH\_HEALTH\_FULL\_LOW & 650 &  Lowest value for undamaged vehicle \\ 
VEH\_HEALTH\_WHITESMOKE & 649 & White smoke from engine \\ 
VEH\_HEALTH\_WHITESMOKE\_LOW & 550 & Lowest value for white engine smoke \\ 
VEH\_HEALTH\_GREYSMOKE & 549 & Grey smoke from engine \\ 
VEH\_HEALTH\_GREYSMOKE\_LOW & 390 & Lowest value for grey engine smoke\\ 
VEH\_HEALTH\_BLACKSMOKE & 389 & Black smoke from engine \\ 
VEH\_HEALTH\_BLACKSMOKE\_LOW & 250 & Lowest value for black engine smoke \\ 
VEH\_HEALTH\_ONFIRE & 249 & Sets car on fire \\
\hline
\end{tabular}
\bigskip
\\You can test this in ISAMPP sandbox game mode with command \textbf{/setvehiclehealth [ID]} while being in any vehicle.
\bigskip
\\Resources: \url{https://open.mp/docs/scripting/resources/vehiclehealth}


\section{Vehicle Colors}
List of all available colors for vehicles in game. Color names using prefix \textit{CARCOL\_SAMP\_} are supported only in SA-MP version 0.3x.
\bigskip
\\Resources: \url{https://open.mp/docs/scripting/resources/vehiclecolorid}


\section{Vehicle Paintjobs}
\begin{sloppypar}
List of all available paintjobs for vehicles in game, starting with prefix \textit{PAINTJOB\_} followed by vehicle name and name of paintjob. Use \textit{PAINTJOB\_REMOVE} to remove applied paintjob.
\end{sloppypar}
\bigskip
\noindent \textbf{Example of implementation:}
\begin{verbatim}
AddStaticVehicle(VEH_CAMPER, 425.7991, 2493.3472, 16.5794,
                 180, CARCOL_WHITE, CARCOL_WHITE);
ChangeVehiclePaintjob(vehicleid, PAINTJOB_CAMPER_TRUTH);
\end{verbatim}
\bigskip
Resources: \url{https://open.mp/docs/scripting/resources/paintjobs}


\section{Weapon Identifiers}
Weapon identifiers are included as macro defines with prefix \textit{WEAP\_}. In ISAMPP sandbox game mode weapon pickups are fully functional. Every weapon in game can be used only for certain weapon slot and weapon identifiers are ordered in file \textit{i\_weaponids.inc} by weapon slot number.
\bigskip
\\Resources: \url{https://open.mp/docs/scripting/resources/weaponids}


\section{Weather Identifiers}
Weather identifiers are included as macro defines with prefix \textit{WEATHER\_}. In ISAMPP sandbox game mode you can test weather settings with command \textbf{/w [ID]}
\bigskip
\\Resources: \url{https://open.mp/docs/scripting/resources/weatherid}


\newpage
\section{Custom Functions}
ISAMPP uses various stock functions which may be useful in creating your own game modes for SA-MP or simply for debugging purposes. These stock functions are defined in \textit{i\_sampp.inc} file.


\subsection{isampp\_console\_printversion()}
Outputs ISAMPP version to server console.
\bigskip
\\\textbf{Example of implementation:}
\begin{verbatim}
main() {
    isampp_console_printversion();
}
\end{verbatim}


\subsection{pawncc\_console\_printversion()}
Outputs version of compiler to server console if Pawncc is used.
\bigskip
\\\textbf{Example of implementation:}
\begin{verbatim}
main() {
    pawncc_console_printversion();
}
\end{verbatim}


\subsection{MppTeleport(playerid, locationid)}
Teleports player to desired location passed as parameter 'locationid'.
\bigskip
\\\textbf{Example of implementation:}
\begin{verbatim}
if (strcmp("/tp barbershop", cmdtext, true, 20) == 0) {
    MppTeleport(playerid, LOC_BARBERSHOP);
    return 1;
}
\end{verbatim}


\subsection{MppTeleportEx(playerid, locationid, pstringcolor)}
Same as MppTeleport, plus outputs location name to in-game chat window. \\Parameter 'pstringcolor' must be in hexadecimal format 0xRRGGBBAA.
\bigskip
\\\textbf{Example of implementation:}
\begin{verbatim}
if (strcmp("/tp barbershop", cmdtext, true, 20) == 0) {
    MppTeleportEx(playerid, LOC_BARBERSHOP, COLOR_LIMEGREEN);
    return 1;
}
\end{verbatim}


\subsection{MppTeleportToCoords(playerid, x, y, z, interiorid, facingangle)}
Teleports player to specified xyz coordinates, supplied with interior identifier and player facing angle.
\bigskip
\\\textbf{Example of implementation:}
\begin{verbatim}
if (strcmp("/tpcoord", cmdtext, true, 20) == 0) {
    MppTeleportToCoords(playerid, 49.4172, 2512.4282, 16.4844, 0, 272);
    return 1;
}
\end{verbatim}


\newpage
\subsection{MppShowPlayerPosition(playerid, pstringcolor)}
Outputs current player location coordinates, interior identifier, facing angle and player camera position coordinates to in-game chat window.
\\Parameter 'pstringcolor' must be in hexadecimal format 0xRRGGBBAA.
\\This function might not work properly if player is in a vehicle.
\bigskip
\\\textbf{Example of implementation:}
\begin{verbatim}
if (strcmp("/showplayerpos", cmdtext, true, 15) == 0) {
    MppShowPlayerPosition(playerid, COLOR_LIGHTRED);
    return 1;
}
\end{verbatim}


\subsection{MppShowVehicleInfo(playerid, vehicleid, pstringcolor)}
Outputs ID, model, health, position and rotation of vehicle in which is player currently sitting to in-game chat window.
\\Parameter 'pstringcolor' must be in hexadecimal format 0xRRGGBBAA.
\\Note: You must pass vehicleid parameter if you want to get model name else function returns '0 Unknown'.
\bigskip
\\\textbf{Example of implementation:}
\begin{verbatim}
new VehicleModelID = 0;

public OnPlayerCommandText(playerid, cmdtext[])
{
    if (strcmp("/showvehicleinfo", cmdtext, true, 15) == 0) {
        MppShowVehicleInfo(playerid, VehicleModelID, COLOR_LIGHTBLUE);
        return 1;
    }
    return 0;
}

public OnPlayerEnterVehicle(playerid, vehicleid, ispassenger)
{
    VehicleModelID = GetVehicleModel(vehicleid);
    return 1;
}

public OnPlayerExitVehicle(playerid, vehicleid)
{
    VehicleModelID = 0;
    return 1;
}
\end{verbatim}


\subsection{MppGetPlayerName(playerid)}
Returns player nick/name from given playerid.
\bigskip
\\\textbf{Example of implementation:}
\begin{verbatim}
if(strcmp(cmdtext, "/myname", true) == 0) {
    SendClientMessage(playerid, COLOR_LIGHTBLUE, MppGetPlayerName(playerid));
    return 1;
}
\end{verbatim}



\newpage
\section{Legacy Functions}
Legacy functions are implemented in file \textit{i\_funcl.inc} for backwards compatibility with really old SA-MP gamemodes which often use them.

\subsection{isnull(string)}
Checks whether a string is equal to null (empty). More efficient than checking if strlen is equal to 0.

\subsection{rot13(string[])}
Rotates the alphabet in string by half of its length - 13 characters. It is a symmetric operation: applying it twice on the same string reveals the original.

\subsection{strcpy(dest[], const source[], len = sizeof(dest))}
Copies the source string to the destination string.

\subsection{strclr(string[])}
Empties and clears given string.

\subsection{strisempty(const string[])}
Returns true if the given string is empty, otherwise returns false.

\subsection{strrest(const string[], \&index)}
Splits string and gives back remaining part of the string divided by space ' ' character as default delimiter.

\subsection{strtok(const string[], \&index)}
Strtok is used for splitting strings and was used as one of the methods for creating game commands with arguments. Strings are divided by space ' ' character as default delimiter.

\subsection{strtolower(string[])}
Changes all characters in the string to lowercase.

\subsection{strtoupper(string[])}
Changes all characters in the string to uppercase.



\newpage
\section{SA-MP Scripting Basics}

\subsection{Glossary}
\begin{tabular}{ |l|l| } 
\hline
Word & Meaning \\
\hline
PAWN & The scripting language used to make SA-MP scripts \\
Gamemode & The main game script that runs on a server \\
Filterscripts & Scripts that run alongside gamemodes \\
Plugins & Extra functions and other features added through .DLL or .SO libraries \\
Include & Pieces of script in .INC files to be included in Filterscripts/Gamemodes \\
Pawno & The script editor most people use for PAWN programming \\
Pawncc & The compiler that compiles .pwn to .amx files \\
Masterlist & The server SA-MP stores its data on such as the Internet list \\
\hline
\end{tabular}
\bigskip
\\Read More: \url{https://open.mp/docs/scripting/resources/glossary}


\subsection{Escape Codes}
\begin{sloppypar}
When creating a string you may find that some character may be impossible or extremely difficult to express in the source code of your script. This is where escape codes come in handy - these allow you to use the symbols and expressions that come under this category. Below is a list of escape codes for the PAWN language.
\end{sloppypar}
\bigskip
\noindent\begin{tabular}{ |l|l| }
\hline
Code & Description \\
\hline
$\backslash a$ or $\backslash 7$ & Audible beep (on server machine) \\
$\backslash b$ & Backspace \\
$\backslash e$ & Escape \\
$\backslash f$ & Form feed \\
$\backslash n$ & New line \\
$\backslash r$ & Carriage return \\
$\backslash t$ & Horizontal tab \\
$\backslash v$ & Vertical tab \\
$\backslash h$ & Backslash \\
$\backslash h'$ & Single quote \\
$\backslash h"$ & Double quote \\
$\backslash \%$ & Percent sign \\
$\backslash ddd;$ & Character code with decimal code \\
$\backslash xhhh;$ & Character code with hexidecimal code \\
\hline
\end{tabular}
\bigskip
\\\textit{\textbf{Note:} The semicolon after the nddd; and xhhh; codes is optional. Its purpose is to give the escape sequence sequence an explicit termination symbol when it is used in a string constant.}
\bigskip
\\Read More: \url{https://open.mp/docs/scripting/resources/escapecodes}


\newpage
\subsection{Limits}
\textbf{In-game Entities}
\bigskip
\\\begin{tabular}{ |l|l| } 
\hline
Type & Limit \\
\hline
Players & 1000 \\
Vehicles & 2000 \\
Vehicle Models & Unlimited \\
Objects & 1000 \\
Virtual Worlds & 2,147,483,647 \\
Interiors & 256 \\
Classes & 320 \\
Map Icons & 100 \\
Race Checkpoints & 1 \\
Checkpoints & 1 \\
Pickups & 4096 \\
Global 3D Labels & 1024 \\
Per-player 3D Text Labels & 1024 \\
Chat Bubble String & 144 \\
SetObjectMaterialText length & 2048 \\
SetPlayerObjectMaterialText length & 2048 \\
Gangzones & 1024 \\
Menus & 128 \\
Attached player objects & 10 \\
Player Variables & 800 \\
Actors (since SA-MP 0.3.7) & 1000 \\
Explosions & 10 \\
\hline
\end{tabular}
\bigskip
\\\textbf{Textdraws}
\bigskip
\\\begin{tabular}{ |l|l| } 
\hline
Type & Limit \\
\hline
String Length & 1024 \\
Shown In A Single Client's Screen & 2048 + 256 \\
Shown In A Single Client's Screen (sprites) & 100 \\
Created Serverwise (Global) & 2048 \\
Created Serverwise (Per-Player) & 256 \\
\hline
\end{tabular}
\bigskip
\\\textbf{Dialogs}
\bigskip
\\\begin{tabular}{ |l|l| } 
\hline
Type & Limit \\
\hline
Dialog IDs & 32768 \\
Info (Main text) & 4096 \\
Caption & 64 \\
Input Text Box & 128 \\
Tab List Columns & 4 \\
Tab List Column Characters & 128 \\
Tab List Row Characters & 256 \\
\hline
\end{tabular}
\bigskip
\\Read More: \url{https://open.mp/docs/scripting/resources/limits}


\newpage
\section{SA-MP Macros}
SA-MP has in certain cases various macros replacing numeric identifiers by default.
\subsection{Animations List}
Read More: \url{https://open.mp/docs/scripting/resources/animations}


\subsection{Bullet Hit Types}
To be used with function \textit{OnPlayerWeaponShot}.
\bigskip
\\\begin{tabular}{ |c|l| } 
\hline
Value & Name \\
\hline
0 & BULLET\_HIT\_TYPE\_NONE \\
1 & BULLET\_HIT\_TYPE\_PLAYER \\
2 & BULLET\_HIT\_TYPE\_VEHICLE \\
3 & BULLET\_HIT\_TYPE\_OBJECT \\
4 & BULLET\_HIT\_TYPE\_PLAYER\_OBJECT \\
\hline
\end{tabular}
\bigskip
\\Read More: \url{https://open.mp/docs/scripting/resources/bullethittypes}


\subsection{Camera Cut Styles}
Camera Cut Styles to be used with functions \textit{SetPlayerCameraLookAt}, \textit{InterpolateCameraPos} and \textit{InterpolateCameraLookAt}.
\bigskip
\\\begin{tabular}{ |c|l| } 
\hline
Value & Name \\
\hline
1 & CAMERA\_MOVE \\ 
2 & CAMERA\_CUT \\ 
\hline
\end{tabular}
\bigskip
\\Read More: \url{https://open.mp/docs/scripting/resources/cameracutstyles}


\subsection{Component Slots}
All available component slots for vehicle customization.\\To be used with function \textit{GetVehicleComponentInSlot(vehicleid, \textbf{slot});}
\bigskip
\\\begin{tabular}{ |c|l| } 
\hline
Value & Name \\
\hline
0 & CARMODTYPE\_SPOILER \\
1 & CARMODTYPE\_HOOD \\
2 & CARMODTYPE\_ROOF \\
3 & CARMODTYPE\_SIDESKIRT \\
4 & CARMODTYPE\_LAMPS \\
5 & CARMODTYPE\_NITRO \\
6 & CARMODTYPE\_EXHAUST \\
7 & CARMODTYPE\_WHEELS \\
8 & CARMODTYPE\_STEREO \\
9 & CARMODTYPE\_HYDRAULICS \\
10 & CARMODTYPE\_FRONT\_BUMPER \\
11 & CARMODTYPE\_REAR\_BUMPER \\
12 & CARMODTYPE\_VENT\_RIGHT \\
13 & CARMODTYPE\_VENT\_LEFT \\
\hline
\end{tabular}
\bigskip
\\Read More: \url{https://open.mp/docs/scripting/resources/Componentslots}


\newpage
\subsection{Connection Status}
\begin{sloppypar}

Connection status returned as value of function \textit{NetStats\_ConnectionStatus(playerid);}
\end{sloppypar}
\bigskip
\noindent\begin{tabular}{ |c|l|l| } 
\hline
Value & Name & Description \\
\hline
0 & NO\_ACTION & N/A \\
1 & DISCONNECT\_ASAP & OnPlayerDisconnect called \\
2 & DISCONNECT\_ASAP\_SILENTLY & N/A \\
3 & DISCONNECT\_ON\_NO\_ACK & N/A \\
4 & REQUESTED\_CONNECTION & Connection request cookie sent \\
5 & HANDLING\_CONNECTION\_REQUEST & N/A \\
6 & UNVERIFIED\_SENDER & N/A \\
7 & SET\_ENCRYPTION\_ON\_MULTIPLE\_16\_BYTE\_PACKET & N/A \\
8 & CONNECTED & playerid is connected \\
\hline
\end{tabular}
\bigskip
\\Read More: \url{https://open.mp/docs/scripting/resources/connectionstatus}


\subsection{Constants List}
Pre-defined constants and identifiers in various SA-MP header files.
\bigskip
\\Read More: \url{https://open.mp/docs/scripting/resources/constants}


\subsection{Dialog Styles}
Styles of dialog windows in SA-MP, to be used with function \textit{ShowPlayerDialog}.
\bigskip
\\\begin{tabular}{ |l|l| } 
\hline
Style & Name \\
\hline
Style 0 & DIALOG\_STYLE\_MSGBOX \\
Style 1 & DIALOG\_STYLE\_INPUT \\
Style 2 & DIALOG\_STYLE\_LIST \\
Style 3 & DIALOG\_STYLE\_PASSWORD \\
Style 4 & DIALOG\_STYLE\_TABLIST \\
Style 5 & DIALOG\_STYLE\_TABLIST\_HEADERS \\
\hline
\end{tabular}
\bigskip
\\Read More: \url{https://open.mp/docs/scripting/resources/dialogstyles}


\subsection{Fighting Styles}
\begin{sloppypar}
Fighting Styles to be used with functions \textit{SetPlayerFightingStyle(playerid, \textbf{style})} and \textit{GetPlayerFightingStyle}.
\end{sloppypar}
\bigskip
\noindent\begin{tabular}{ |c|l| } 
\hline
Value & Name \\
\hline
4 & FIGHT\_STYLE\_NORMAL \\
5 & FIGHT\_STYLE\_BOXING \\
6 & FIGHT\_STYLE\_KUNGFU \\
7 & FIGHT\_STYLE\_KNEEHEAD \\
15 & FIGHT\_STYLE\_GRABKICK \\
16 & FIGHT\_STYLE\_ELBOW \\
\hline
\end{tabular}
\bigskip
\\Read More: \url{https://open.mp/docs/scripting/resources/fightingstyles}


\newpage
\subsection{Keys}
\begin{sloppypar}
To be used with functions \textit{GetPlayerKeys} and textit{OnPlayerKeyStateChange}.
\end{sloppypar}
\bigskip
\noindent\begin{tabular}{ |c|l| } 
\hline
Value & Name \\
\hline
1 & KEY\_ACTION \\
2 & KEY\_CROUCH \\
4 & KEY\_FIRE \\
8 & KEY\_SPRINT \\
16 & KEY\_SECONDARY\_ATTACK \\
32 & KEY\_JUMP \\
64 & KEY\_LOOK\_RIGHT \\
128 & KEY\_HANDBRAKE/KEY\_AIM \\
256 & KEY\_LOOK\_LEFT \\
512 & KEY\_LOOK\_BEHIND \\
512 & KEY\_SUBMISSION \\
1024 & KEY\_WALK \\
2048 & KEY\_ANALOG\_UP \\
4096 & KEY\_ANALOG\_DOWN \\
8192 & KEY\_ANALOG\_LEFT \\
16384 & KEY\_ANALOG\_RIGHT \\
65536 & KEY\_YES \\
131072 & KEY\_NO \\
262144(4) & KEY\_CTRL\_BACK \\
- & UNDEFINED \\
-128 & KEY\_UP \\
128 & KEY\_DOWN \\
-128 & KEY\_LEFT \\
128 & KEY\_RIGHT \\
\hline
\end{tabular}
\bigskip
\\Read More: \url{https://open.mp/docs/scripting/resources/keys}


\subsection{Map Icon Styles}
To be used with function \textit{SetPlayerMapIcon}.
\bigskip
\\\begin{tabular}{ |c|l|l|l| } 
\hline
Value & Name & Marker & Radar Map Range \\
\hline
0 & MAPICON\_LOCAL & No & Close proximity only \\
1 & MAPICON\_GLOBAL & No & Show on radar edge as long as in range \\
2 & MAPICON\_LOCAL\_CHECKPOINT & Yes & Close proximity only \\
3 & MAPICON\_GLOBAL\_CHECKPOINT & Yes & Show on radar edge as long as in range \\
\hline
\end{tabular}
\bigskip
\\Read More: \url{https://open.mp/docs/scripting/resources/mapiconstyles}


\subsection{Marker Modes}
To be used with function \textit{ShowPlayerMarkers(\textbf{mode});}
\bigskip
\\\begin{tabular}{ |c|l| } 
\hline
Value & Name \\
\hline
0 & PLAYER\_MARKERS\_MODE\_OFF \\ 
1 & PLAYER\_MARKERS\_MODE\_GLOBAL \\ 
2 & PLAYER\_MARKERS\_MODE\_STREAMED \\
\hline
\end{tabular}
\bigskip
\\Read More: \url{https://open.mp/docs/scripting/resources/markermodes}


\newpage
\subsection{Material Text Alignments}
To be used with function \textit{SetObjectMaterialText}.
\bigskip
\\\begin{tabular}{ |c|l| }
\hline
Value & Name \\
\hline
0 & OBJECT\_MATERIAL\_TEXT\_ALIGN\_LEFT \\
1 & OBJECT\_MATERIAL\_TEXT\_ALIGN\_CENTER \\
2 & OBJECT\_MATERIAL\_TEXT\_ALIGN\_RIGHT \\
\hline
\end{tabular}
\bigskip
\\Read More: \url{https://open.mp/docs/scripting/resources/materialtextalignment}


\subsection{Material Text Sizes}
To be used with function \textit{SetObjectMaterialText}.
\bigskip
\\\begin{tabular}{ |c|l| }
\hline
Value & Name \\
\hline
10 & OBJECT\_MATERIAL\_SIZE\_32x32 \\
20 & OBJECT\_MATERIAL\_SIZE\_64x32 \\
30 & OBJECT\_MATERIAL\_SIZE\_64x64 \\
40 & OBJECT\_MATERIAL\_SIZE\_128x32 \\
50 & OBJECT\_MATERIAL\_SIZE\_128x64 \\
60 & OBJECT\_MATERIAL\_SIZE\_128x128 \\
70 & OBJECT\_MATERIAL\_SIZE\_256x32 \\
80 & OBJECT\_MATERIAL\_SIZE\_256x64 \\
90 & OBJECT\_MATERIAL\_SIZE\_256x128 \\
100 & OBJECT\_MATERIAL\_SIZE\_256x256 \\
110 & OBJECT\_MATERIAL\_SIZE\_512x64 \\
120 & OBJECT\_MATERIAL\_SIZE\_512x128 \\
130 & OBJECT\_MATERIAL\_SIZE\_512x256 \\
140 & OBJECT\_MATERIAL\_SIZE\_512x512 \\
\hline
\end{tabular}
\bigskip
\\Read More: \url{https://open.mp/docs/scripting/resources/materialtextsizes}


\subsection{Object Edition Response Types}
To be used with functions \textit{OnPlayerEditObject} and \textit{OnPlayerEditAttachedObject}.
\bigskip
\\\begin{tabular}{ |c|l|l| }
\hline
Value & Name & Description \\
\hline
0 & EDIT\_RESPONSE\_CANCEL & Player cancelled (ESC) \\
1 & EDIT\_RESPONSE\_FINAL & Player clicked on save \\
2 & EDIT\_RESPONSE\_UPDATE & Player moved the object (edition did not stop) \\
\hline
\end{tabular}
\bigskip
\\Read More: \url{https://open.mp/docs/scripting/resources/objecteditionresponsetypes}


\newpage
\subsection{Player States}
To be used with \textit{GetPlayerState} function or \textit{OnPlayerStateChange} callback.
\bigskip
\\\begin{tabular}{ |c|l|l| }
\hline
Value & Name & Description \\
\hline
0 & PLAYER\_STATE\_NONE & Default state, used while initializing \\
1 & PLAYER\_STATE\_ONFOOT & Player is on foot \\
2 & PLAYER\_STATE\_DRIVER & Player is driving a vehicle \\
3 & PLAYER\_STATE\_PASSENGER & Player is in a vehicle as a passenger \\
4 & PLAYER\_STATE\_EXIT\_VEHICLE & Player is exiting vehicle \\
5 & PLAYER\_STATE\_ENTER\_VEHICLE\_DRIVER & Entering vehicle (driver) \\
6 & PLAYER\_STATE\_ENTER\_VEHICLE\_PASSENGER & Entering vehicle (passenger) \\
7 & PLAYER\_STATE\_WASTED & Player is either dead or in class selection \\
8 & PLAYER\_STATE\_SPAWNED & Player just spawned \\
9 & PLAYER\_STATE\_SPECTATING & Player is in spectator mode \\
\hline
\end{tabular}
\bigskip
\\Read More: \url{https://open.mp/docs/scripting/resources/playerstates}


\subsection{Pvar Types}
Types of player variables (also called pvar types) used in Per-player variable system.
\bigskip
\\\begin{tabular}{ |c|l| }
\hline
Value & Name \\
\hline
0 & PLAYER\_VARTYPE\_NONE \\
1 & PLAYER\_VARTYPE\_INT \\
2 & PLAYER\_VARTYPE\_STRING \\
3 & PLAYER\_VARTYPE\_FLOAT \\
\hline
\end{tabular}
\bigskip
\\Read More: \url{https://open.mp/docs/scripting/resources/pvartypes}


\subsection{Record Types}
To be used with \textit{StartRecordingPlayerData} function.
\bigskip
\\\begin{tabular}{ |c|l| }
\hline
Value & Name \\
\hline
0 & PLAYER\_RECORDING\_TYPE\_NONE \\
1 & PLAYER\_RECORDING\_TYPE\_DRIVER \\
2 & PLAYER\_RECORDING\_TYPE\_ONFOOT \\
\hline
\end{tabular}
\bigskip
\\Read More: \url{https://open.mp/docs/scripting/resources/recordtypes}


\subsection{Select Object Types}
Select object types used by \textit{OnPlayerSelectObject} function.
\bigskip
\\\begin{tabular}{ |c|l| }
\hline
Value & Name \\
\hline
1 & SELECT\_OBJECT\_GLOBAL\_OBJECT \\
2 & SELECT\_OBJECT\_PLAYER\_OBJECT \\
\hline
\end{tabular}
\bigskip
\\Read More: \url{https://open.mp/docs/scripting/resources/selectobjecttypes}


\newpage
\subsection{Shop Names}
To be used with \textit{SetPlayerShopName(playerid, \textbf{shopname[]})} function.
\bigskip
\\\begin{tabular}{ |l|l|l| }
\hline
Name & Description & ISAMPP Location ID \\
\hline
FDPIZA & Stock Pizza Stack interior & LOC\_PIZZASTACK \\
FDCHICK & Stock Cluckin' Bell interior & LOC\_CBELL \\
FDBURG & Stock Burger Shot interior & LOC\_BURGERSHOT \\
AMMUN1 & First Ammu-Nation interior & LOC\_AMMUNATION2 \\
AMMUN2 & Second Ammu-Nation interior & LOC\_AMMUNATION3 \\
AMMUN3 & Third Ammu-Nation interior & LOC\_AMMUNATION4 \\
AMMUN4 & Fourth Ammu-Nation interior & LOC\_AMMUNATION \\
AMMUN5 & Fifth Ammu-Nation interior & LOC\_AMMUNATION5 \\
\hline
\end{tabular}
\bigskip
\\Read More: \url{https://open.mp/docs/scripting/resources/shopnames}


\subsection{Special Actions}
Used by \textit{GetPlayerSpecialAction} and \textit{SetPlayerSpecialAction} functions.
\bigskip
\\\begin{tabular}{ |c|l|l| }
\hline
Value & Action & Description \\
\hline
0 & SPECIAL\_ACTION\_NONE & Clears player of special actions \\
1 & SPECIAL\_ACTION\_DUCK & Detect if the player is crouching \\
2 & SPECIAL\_ACTION\_USEJETPACK & Make the player using jetpack \\
3 & SPECIAL\_ACTION\_ENTER\_VEHICLE & Player is entering a vehicle \\
4 & SPECIAL\_ACTION\_EXIT\_VEHICLE & Player is exiting a vehicle \\
5 & SPECIAL\_ACTION\_DANCE1 & Applies dancing animation for player \\
6 & SPECIAL\_ACTION\_DANCE2 & Applies dancing animation for player \\
7 & SPECIAL\_ACTION\_DANCE3 & Applies dancing animation for player \\
8 & SPECIAL\_ACTION\_DANCE4 & Applies dancing animation for player \\
10 & SPECIAL\_ACTION\_HANDSUP & Make the player put hands up \\
11 & SPECIAL\_ACTION\_USECELLPHONE & Make the player speaking on cellphone \\
12 & SPECIAL\_ACTION\_SITTING & Detects if the player is sitting \\
13 & SPECIAL\_ACTION\_STOPUSECELLPHONE & Makes players stop using cellphone \\
20 & SPECIAL\_ACTION\_DRINK\_BEER & Icrease the player's drunk level \\
21 & SPECIAL\_ACTION\_SMOKE\_CIGGY & Give the player a cigar \\
22 & SPECIAL\_ACTION\_DRINK\_WINE & Give the player a wine bottle \\
23 & SPECIAL\_ACTION\_DRINK\_SPRUNK & Give the player a sprunk bottle \\
24 & SPECIAL\_ACTION\_CUFFED & Force the player in to cuffs \\
25 & SPECIAL\_ACTION\_CARRY & Apply a carrying animation to the player \\
68 & SPECIAL\_ACTION\_PISSING & Player performs the pissing animation \\
\hline
\end{tabular}
\bigskip
\\Read More: \url{https://open.mp/docs/scripting/resources/specialactions}


\subsection{Spectate Modes}
Used by \textit{PlayerSpectatePlayer} and \textit{PlayerSpectateVehicle} functions.
\bigskip
\\\begin{tabular}{ |l|l| }
\hline
Name & Description \\
\hline
SPECTATE\_MODE\_NORMAL & Normal spectate mode - third person camera \\
SPECTATE\_MODE\_FIXED & Used with SetPlayerCameraPos function \\
SPECTATE\_MODE\_SIDE & Camera will be attached to the side of the player/vehicle \\
\hline
\end{tabular}
\bigskip
\\Read More: \url{https://open.mp/docs/scripting/resources/spectatemodes}


\newpage
\subsection{Starting IDs}
Everything like objects, players or vehicles use IDs. Some IDs start with 0, others start with 1. If you plan to use an array to hold all IDs you might have to subtract 1 to get the array element ID.
\bigskip
\\\begin{tabular}{ |c|l| }
\hline
Starting ID & Type \\
\hline
0 & 3D Text Label \\
0 & Actor \\
0 & File \\
0 & GangZone \\
1 & Object \\
0 & Pickup \\
0 & Player \\
0 & Player Class \\
0 & TextDraw / PlayerTextDraw \\
1 & Timer \\
1 & Vehicle \\
\hline
\end{tabular}
\bigskip
\\Read More: \url{https://open.mp/docs/scripting/resources/startingids}


\subsection{Svar Types}
Used by \textit{GetSVarType(\textbf{varname})} function.
\bigskip
\\\begin{tabular}{ |c|l| }
\hline
Value & Name \\
\hline
0 & SERVER\_VARTYPE\_NONE \\
1 & SERVER\_VARTYPE\_INT \\
2 & SERVER\_VARTYPE\_STRING \\
3 & SERVER\_VARTYPE\_FLOAT \\
\hline
\end{tabular}
\bigskip
\\Read More: \url{https://open.mp/docs/scripting/resources/svartypes}



\end{document}
